\thispagestyle{plain}

The analysis of the effects of heterogeneity on aggregate economic outcomes has 
seen a resurgence in the recent macroeconomic literature. The exponential increase
 in computer power over the last decades has allowed researchers to solve ever 
more complex theoretical models with meaningful heterogeneity along various 
dimensions, while at the same time bringing ever more granular micro-level data 
to the table when testing the model predictions. \\
This thesis explores two varieties of this recent vintage of models of heterogeneity.
The first part of the thesis investigates the implications for wealth distributions
of combining the standard life-cycle incomplete markets model of household 
consumption with income processes featuring heterogeneity in individual-specific
growth rates, which households can learn about over the course of their working life.
To this extent, first the recent literature on partial insurance and models
of wealth inequality is reviewed. Then, income processes with profile heterogeneity
 are estimated from PSID and BHPS data. The results confirm the findings of 
previous studies that allowing for profile heterogeneity significantly lowers
the estimated persistence and innovation variance of persistent shocks to household
income, and documents substantial variation in the estimated parameters of the 
income process across time periods and measures of household income.
The estimated income processes obtained are then used in a quantitative model of 
household consumption and saving in order to investigate the implications for the 
model predictions on the wealth distribution. The model is calibrated to
 empirical wealth distributions obtained from the SCF and the BHPS, and it is shown
that the inclusion of individual-specific growth rate heterogeneity in income 
severely deteriorates the models ability to fit the shape of the data. 
Comparative statics exercises are performed to identify the drivers in the models
failure to match the empirical profile of wealth holdings, which show that it 
is precisely the two key parameters which differ between the standard AR(1) model
and the heterogeneous profile model, the persistence and variance of the permanent
shock to household income, which drive model fit. 
The second part of the thesis looks at heterogeneity on the production side of 
the economy and its implications for international trade. Following an existing 
approach in the literature, we develop testable implications of the Melitz and 
Ottaviano (2008) model of trade, in which firms differ in their 
productivity and have to make production and exporting decisions in the face of 
costs to trade. Applying an estimation strategy previously used in the literature,
we find weak support of the models predictions in data for 64 manufacturing industries
in the NAFTA member countries Canada, Mexico and USA. We then test additional 
model predictions by constructing a measure of entry conditions by industry based
on firm turnover, which allows us to divide our sample into fixed and free entry
industries. Furthermore, we include the effects of third country tariff barriers
on the relative performance of two trading partners' industries. While the results
are broadly in line with model predictions, we find some evidence of violations
of the predictions in the data.