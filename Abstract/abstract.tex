\thispagestyle{plain}

The analysis of the effects of heterogeneity on aggregate economic outcomes has 
seen a resurgence in the recent macroeconomic literature. The exponential increase
 in computer power over the last decades has allowed researchers to solve ever 
more complex theoretical models with meaningful heterogeneity along various 
dimensions, while at the same time bringing ever more granular micro-level data 
to the table when testing the model predictions. \\
This thesis explores two varieties of this recent vintage of models of heterogeneity.
The first part of the thesis investigates the implications for wealth distributions
of combining the standard life-cycle incomplete markets model of household 
consumption with income processes featuring heterogeneity in individual-specific
growth rates, which households can learn about over the course of their working life.
To this extent, first the recent literature on partial insurance and models
of wealth inequality is reviewed. Then, income processes with profile heterogeneity
 are estimated from PSID and BHPS data. Finally, the estimated income processes
are used in a quantitative model of household consumption and saving, which is 
calibrated to empirical wealth distributions obtained from the SCF and the BHPS.
Comparative statics exercises are performed to identify the drivers in the models
success (or failure) to match the empirical profile of wealth holdings.
The second part of the thesis looks at heterogeneity on the production side of 
the economy and its implications for international trade. Following an existing 
approach in the literature, we develop testable implications of the Melitz and 
Ottaviano (2008) model of trade, in which firms differ in their 
productivity and have to make production and exporting decisions in the face of 
costs to trade. Our approach allows us to test the effects of NAFTA on productivity 
in 64 manufacturing sectors in North America and thereby complement and extend
 the existing literature on the effects of trade liberalisations.