\section{Motivation}
Distributional questions are increasingly making a comeback in economics. 


\section{Outline}
This thesis is structured as follows. Chapter 2 gives an overview of the research on theoretical models of the household wealth distribution in the last two decades. It will highlight the key empirical challenges by presenting stylized facts on the cross sectional wealth distribution in a number of countries and their evolution over time. Then, an overview of existing modeling approaches is presented,  

Chapter 3 builds on the discussion in Chapter 2 by constructing a model of learning about idiosyncratic income processes and using it to simulate wealth distributions. As a first step, income processes with profile heterogeneity are estimated for different sub-periods of PSID data from 1968 to 1999 and from BHPS data in order to assess the stability of the cross-sectional variance of income growth rates across time and labor markets. The estimates are then used as inputs in a structural model of household saving first employed by \citet{Guvenen2007}, with the addition of both structural breaks in the cross-sectional variance of growth rates and habits in consumption. The models predictions for the evolution of the US consumer wealth distribution are then benchmarked against data from the Survey of Consumer Finances. 

Chapter 4 gives a brief introduction into trade models based on firm-level heterogeneity, before developing an estimable model in the spirit of \citet{Chen2009}. The model is then tested on a data set of prices, productivity, and markups for nine manufacturing industries in the Canada, Mexico and the United States over the period of 1994 to 2010. In an extension of the approach of \citet{Chen2009}, we also conduct a sub-sample analysis in which we split the same into fixed and free entry industries, based on a measure of firm turnover developed on the basis of prior research. 

Chapter 5 concludes the thesis and discusses potential avenues for future research.