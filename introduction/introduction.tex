\section{Motivation}
Distributional questions are increasingly making a comeback in economics. In spite
of the famous -- or infamous -- warning of \citet{Lucas2004} that the focus 
on questions of distribution is one of the most "seductive (...) and poisoning"
tendencies in economics\footnote{In fairness it has to be said that Lucas' quote
is often taken out of context, as he was not actually advising against studying 
distributional issues entirely, but merely pointing out that economic growth has
played a much more important role in raising people out of poverty than re-
distribution of current resources at any point in time could have achieved.},
many fields of economics that have long relied on simplistic models of representative
households and firms have increasingly taken the issue of modelling heterogeneity
across economic agents seriously. At the very least since the Great Recession
triggered by the financial crisis of 2007--2008, issues of distribution have also
taken centre stage in the public economic discourse. Work on the increase in 
income inequality, especially at the top end of the income distribution, and 
the rising inequality in wealth holdings in advanced economies has played a 
prominent role in the public debate in recent years, the most prominent recent
example being \citet{Piketty2014}, a rare instance of a book largely based on 
economic scholarship being widely discussed and sold (if maybe not read) by
a mainstream audience. However, while the public has only recently started to 
take an interest in issues of inequality and distribution, the economic literature
has been developing quantitative models of heterogeneity for almost three decades.
Seminal papers such as \citet{Imrohoroglu1989}, \citet{Huggett1993}, and
\citet{Aiyagari1994} have laid the groundwork for a vast literature explicitly 
modelling the choices of heterogeneous agents based on microeconomic evidence.
A major factor in the move towards models of explicit heterogeneity have been 
the huge advancements in computer power in recent decades. With Moore's law
still holding to this day, the transistor count of the fastest microprocessor
today is about two-thousand times as high as that of the fastest microprocessor
twenty-five years ago, when \citet{Zeldes1989} published one of the first works
that numerically solved a household savings problem under uncertainty numerically.
Besides enabling researchers to numerically solve ever more complex optimisation
problems with state spaces of ever more dimensions, the advancements in computer
power have also vastly improved data processing capabilities, a development that
in turn has led to a surge in empirical work exploiting large microeconomic data
sets, the results of which can then be used to validate models of household
and firm behaviour. \\
The present work explores heterogeneity in two different classes of economic models,
using both approaches outlined above. While the first part will build up towards
a quantitative theoretical model of the wealth distribution, the merits of which
will be evaluated against micro survey data, the second part will take the predictions
of a micro-founded model of international trade and test them on data of firm 
behaviour disaggregated at the industry-level. 

\section{Outline}
This thesis is structured as follows. Chapter \ref{ch:wealthmodels} gives an overview of the research 
on theoretical models of the household wealth distribution in the last two decades. 
It will highlight the key empirical challenges by presenting stylised facts on 
the cross sectional wealth distribution using the most recent wave of the UK
Wealth and Asset Survey as an example. Then, the workhorse model of the literature
on household savings decision, the incomplete markets life-cycle model of consumption,
is briefly reviewed along with recent work partial insurance in this model. 
Then, extensions of the model which help it match the stylised facts of empirical
wealth distributions are reviewed.

Chapter \ref{ch:income} builds on the discussion in Chapter \ref{ch:wealthmodels} 
by estimating income processes with profile heterogeneity for different sub-periods 
of PSID data from 1968 to 2013 and from BHPS 
data in order to assess the stability of the cross-sectional variance of income 
growth rates across time and income measures. The estimates are then used
in chapter \ref{ch:learning} as inputs 
in a structural model of household saving first employed by \citet{Guvenen2007}, 
which is calibrated to the empirical wealth distribution using a minimum distance
estimator. After discussing the model fit, comparative statics exercises are 
performed on all parameters of the income process, the upper and lower bounds of
which are taken from the universe of estimation results from chapter \ref{ch:income},
to understand which parameters are most important for the model fit. 

Chapter \ref{ch:trade} gives a brief introduction into trade models based on firm-level 
heterogeneity, before developing an estimable model in the spirit of \citet{Chen2009}. 
The model is then tested on a data set of prices, productivity, and markups for 
64 manufacturing industries in the Canada, Mexico and the United States over 
the period of 1988 to 2010. In an extension of the approach of \citet{Chen2009},
 a sub-sample analysis is conducted in which the observations are split into fixed and 
free entry industries, based on a measure of firm turnover developed on the basis 
of prior research. 

Chapter 5 concludes the thesis and discusses potential avenues for future research.