\section{Introduction}

Monetary policy, forecasting and derivative pricing are a handful of the many reasons that have sparked an interest in bond yields. Most modern economies utilize the term structure of interest rates to conduct monetary policy. Particularly, the short end of the yield curve is exploited to drive changes in the medium and long end of the curve. Focus is drawn to medium and long term yields due to their inherent association with borrowing costs and consequently their tight link to the economy's aggregate demand. Current yield curves bear informational content on future curves and economic activity, rendering them a potent tool for forecasting. Additionally, the valuation of complex financial instruments is often determined through interest rate models. However, despite the fact that bond prices are typically observed, bond yields need to be extrapolated by these bond prices and as a consequence, the estimation of term structure models of interest rates has spawned a wide literature due to its importance to policymakers, academics and practitioners. 

Bonds, unlike other financial assets and macroeconomic variables, enjoy the peculiarity of having many observed yields associated with different maturities, at every given point in time, thus rendering both their time series and cross-sectional properties of interest. An analysis ignoring cross-sectional restrictions is possible, when focusing on a particular segment of the yield curve. However, the incorporation of cross-sectional restrictions comes with its own benefits. First and foremost, the imposition of no-arbitrage restrictions allows the extraction of risk premia by alleviating the difficulty that usually arises, that is, the inability to disentangle risk premia from expectations. Accounting for no-arbitrage introduces an additional probability measure to the physical one, known as the risk-neutral measure. By computing the difference between those two measures, one is capable to obtain estimates of the risk premium. It is important to note that the assumption of no-arbitrage is well grounded given the highly liquid nature of bond markets. In addition, these restrictions further enhance the consistency of yields across time and maturities and improve out-of-sample forecasts by reducing the number of parameters to be estimated within the model. 

Having addressed the importance of working on a set of yields that vary across time and maturities, multivariate models are perceived as the appealing paradigm to capture yield dynamics. A natural response is to consider an unrestricted vector autoregression model. However, the latter is paired with the disadvantage of losing degrees of freedom due to the high-dimensionality of the model. At this point, the advantages of cross-sectional restrictions enter into play by allowing a low-dimensional factor structure to approximate the high-dimensional system. A factor structure appears to be sufficient to be able to replicate all possible shapes of the yield curve. Specifically, yield curves take different forms across time, from U-shaped curves, all the way to flat, upward or downward sloping curves. Nonetheless, typical stylized facts of yield curves include the notion that yields ought to increase with maturity, thus rendering upward sloping curves more characteristic. This fact enhances the liquidity preference theory, which stipulates that a time-varying term premium is required on long term yields to compensate for their relative lack of liquidity. Yields are also known to be highly persistent, as indicated by their strong autocorrelations. An additional trait of the yield curve is the fact that its short end is typically more volatile than its long end. This last stylized fact becomes of particular interest in today's economy, with unconventional monetary policy strategies driving short yields near their zero lower bound. By anchoring the short end of the curve, the volatility has been seen to pick up in the long end of the curve and inversely decrease in the short end. These very stylized facts aid in imposing the restrictions necessary to achieve the factor structure. 

Reaching a consensus that a low-dimensional factor structure has the ability to summarize a complex and high-dimensional structure, the econometrician is now faced with a wide choice of factor structures. At this stage, it is important to note that it is widely accepted, in the literature, that three factors are typically considered sufficient (see \cite{litterman_1991}, \cite{piazzesi_2003}). The choice of factor structures can be synthesized in the following list of alternative models: principal components, interpolation methods and term structure models. In this chapter, arguments are made in support of the latter alternative, as it not only encompasses consistency of yield dynamics through the imposition of no-arbitrage, but it further allows the dissociation of risk premia from expectations' estimates. This chapter, thus, resumes in clarifying the ties yield curve models may have with financial and economic variables, including exchange rates, inflation and growth. This segment builds the necessary grounds for the following chapters, which apply affine term structure models of interest rates to monetary finance applications, with the aim to extract risk premia estimates. 

This chapter benefits from the work of \cite{piazzesi_2010} and \cite{diebold_book}, and is constructed as follows. In the second section, the basic concepts surrounding bond yields and prices are tackled. In the third section, affine term structure models are introduced. The fourth section includes a brief account of the recent developments within this literature, known as macro-finance models. The fifth section provides conclusive remarks. 


\section{Bond prices and yields}

This section establishes the main definitions revolving around term structure modeling. It is important to note that term structure models focus on specific bonds, namely zero-coupon bonds. Those pay no coupons and only pay a single payoff at maturity, known as the face value of the bond, which for simplicity is assumed to amount to 1 unit of currency. Zero-coupon bonds are characterized by being purchased at discount and by the fact that they are considered as default free. Let \(P_t(\tau)\) denote the price of a bond at time t that matures in \(\tau\) periods and \(y_t(\tau)\) denote the yield to maturity, compounded continuously. The following relationship holds.
\begin{equation}
P_t\left(\tau\right)=exp\left[-\tau y_t(\tau)\right] \label{eq: bond_price_ch1}
\end{equation}  

Yields to maturity, also known as zero coupon yields, are thus naturally implied by zero coupon bond prices as follows.
\begin{equation}
y_t\left(\tau\right)=-\frac{log P_t\left(\tau\right)}{\tau} \label{eq: bond_yield_ch1}
\end{equation}

Yields can also be expressed as an average of forward rates, which are the increment observed in the yield for prolonging the maturity by one additional period. The relationship of zero coupon yields and forward rates, in continuous time, is given below.
\begin{equation}
y_t\left(\tau\right)=\frac{1}{\tau}\int^{\tau}_0f_udu \label{eq: bond_yield_forward_ch1}
\end{equation} 

In addition, by combining equations \ref{eq: bond_yield_ch1} and \ref{eq: bond_yield_forward_ch1}, the forward rate curve can be extracted by using the formula below.
\begin{equation}
f_t\left(\tau\right)=-\frac{P'_t\left(\tau\right)}{P_t\left(\tau\right)} \label{eq: forward_ch1}
\end{equation}

where \(P'_t\left(\tau\right)\) designates the first derivative of the bond price \(P_t\left(\tau\right)\).
It is interesting to note that out of the three variables in question, \(P(\tau)\), \(y(\tau)\) and \(f(\tau)\), only one of them suffices to derive the remaining two. 

As previously mentioned, bond yields are not observed and need to be extracted by transforming observed bond prices. Many approaches have been taken across the years. One of them consists of the use of spline methods, including polynomial splines and exponential splines, to name a few. These were deemed dated due to their incapacity to ensure positive forward rates. \cite{fama_bliss_1987} elaborate on this flaw by derive the yield curve using forward rates. This very method is typically used to obtain what are known as unsmoothed Fama-Bliss forward rates. The preponderance of central banks often use interpolation methods, such as the Nelson-Siegel or Svensson method, on those unsmoothed yields, in order to smoothen them.
Factor models have become increasingly popular in the estimation of term structure models as they reduce the dimensionality of the problem whilst enabling the replication of all possible shapes of the yield curve. The most widespread factor designs in term structure modeling are broadly segregated into three families. The first factor structure stems from a principal component analysis, which by construction imposes factors to remain orthogonal whilst factor loadings are left unconstrained. A second structure involves interpolation methods that fit empirical yield curves. Unlike the previous method, factors remain unconstrained and factor loadings are the ones that inherit a particular empirical structure. The third category is known as the no-arbitrage dynamic term structure model. This method imposes restrictions on both factors and loadings. The most important trait of this structure revolves around the imposition of no-arbitrage restrictions on the factor loadings. Although this last class of models is very broad, the most noteworthy subclass is known to be affine term structure models. The next section comprises of a brief account of affine models. 


\section{Affine term structure models}

The pricing of bonds necessitates an equivalent probability measure to the physical one \(\mathbb{P}\), known as the risk-neutral probability measure, denoted by \(\mathbb{Q}\). The very introduction of a second probability measure allows the imposition of the absence of arbitrage opportunities, which according to \cite{almeida_2008}, enhances estimation and forecasting efficiency as well as solidifies the consistency of the model. Assuming no-arbitrage, a bond, that pays a payoff \(\Pi(T)\) at time T, is priced under the physical measure \(\mathbb{P}\) using a pricing kernel M(t). The current price \(\Pi(t)\) is thus the expectation of the discounted future cash flows, as seen below, where \(E^\mathbb{P}_t\) denotes the expectation at time t under the physical measure.
\begin{equation}
\Pi(t)=E^\mathbb{P}_t\left[\frac{M(T)}{M(t)}\Pi(T)\right] \label{eq: noarbitrage_ch1}
\end{equation} 
 
Assuming the kernel dynamics given in equation \ref{eq: kernel_ch1}, where \(\Gamma(t)\) and W(t) represent, respectively, the price of risk and a standard Brownian motion, the two measures, \(\mathbb{P}\) and \(\mathbb{Q}\), are linked through the Radon-Nikodym derivative given in equation \ref{eq: radon_ch1}.
\begin{align}
\frac{dM(t)}{M(t)}&=-r(t)dt-\Gamma(t)'dW(t) \label{eq: kernel_ch1}\\
\frac{d\mathbb{Q}}{d\mathbb{P}}&=exp\left[-\frac{1}{2}\int^T_t\Gamma(s)'\Gamma(s)ds-\int^T_t\Gamma(s)dWs\right] \label{eq: radon_ch1}
\end{align}  

It follows that equation \ref{eq: noarbitrage_ch1} is transformed as shown below.
\begin{equation}
\Pi(t)=E^\mathbb{Q}_t\left[exp\left(-\int^T_tr_udu \: \Pi(T)\right)\right] \label{eq: bond_price_yield_ch1}
\end{equation}  

Let T denote the maturity of a zero-coupon bond that pays one unit of currency at maturity and \(\tau=T-t\) designate the time to maturity. The instantaneous rate, denoted by \(r_t\), is given by the limit of yields \(y_t(\tau)\) as time t tends to T and the bond price is given as follows. 
\begin{equation}
P_t\left(\tau\right)=E^\mathbb{Q}_t\left[exp\left(-\int^T_tr_udu\right)\right] \label{eq: zcb_price_yield_ch1}
\end{equation}  

It is clearly reflected in equation \ref{eq: zcb_price_yield_ch1} that there are two key components to modeling the yield curve, those being the existence of an equivalent measure \(\mathbb{Q}\) to the physical measure \(\mathbb{P}\) and the dynamics of the instantaneous rate \(r_t\) under \(\mathbb{Q}\).
In affine term structure models the dynamics of the instantaneous rate \(r_t\) under \(\mathbb{Q}\) ought to be an affine function of the state variable \(X_t\), which itself is an affine diffusion under the risk-neutral probability measure. The state dynamics follow an affine diffusion process, provided below: 
\begin{equation}
dX_t=\mu(X_t)dt+\sigma(X_t)dW(t) \label{eq: state_ch1}\\
\end{equation}  

where the drift \(\mu(X_t)\) and the variance-covariance matrix \(\sigma(X_t)\sigma(X_t)'\) are affine in \(X_t\).
The drift of the state dynamics takes the following form, \(\mu(X_t)=\kappa(\theta-X_t)\), where \(\kappa\) is the mean reversion matrix and \(\theta\) represents the unconditional mean. As for the diffusion of the process, it takes the following form, \(\sigma(X_t)=\Sigma s(X_t)\), where \(s(X_t)\) is equal to the identity matrix in Gaussian affine models and is a diagonal matrix, of the form \(s_{ii}(X_t)=\sqrt{s_{0,ii}+s_{1,ii}'X_t}\), in the stochastic volatility class of models. More is said on the latter models, given chapter 3 focuses on an exchange rate application of an affine term structure model with stochastic volatility.

Bond prices thus inherit an exponentially affine representation, which is the solution of a system of ordinary differential equations (ODE). These ODE have a closed-form solution when the model is Gaussian and are solved numerically when the model encompasses stochastic volatility.  

It is important to note that Gaussian affine models do not preclude interest rates from being negative. This issue is not of particular interest when interest rates are at a safe distance of the zero lower bound. However, with recent economic developments, interest rates have plummeted to unprecedented levels, sparking thus the need to impose the non-negativity of interest rates. Three different classes of models have been developed to accommodate this situation: shadow rate models, Cox-Ingersoll-Ross models and quadratic models. Quadratic models as in \cite{ahn_2002} are, nonetheless, unable to conform to prolonged periods of zero or near zero interest rates. Conversely, shadow rate models are able to cope with extended periods of near zero rates by rendering instantaneous rates non-linear. Chapter 4 elaborates on the particularity of estimating rates in the vicinity of zero and builds an inflation application of both a Gaussian affine term structure model and a shadow model.

It is worth noting that affine models, despite their advantages in precluding arbitrage opportunities and obtaining known expressions for term premia, come at the disadvantage of being hard to estimate and interpret. More specifically, common issues that arise are the inability to interpret intuitively the latent factors and the global optimum problem.


\section{Macro-finance extensions}

The two previous sections have established that term structure models are of importance to model the dynamics of yields across both their cross-section and time series and are particularly interesting tools due to their simplicity in extending them to more complex and complete frameworks. It has long been instilled that the state of the economy has an impact on financial variables. A clear example of macroeconomic variables feeding financial variables is the effect of the level of inflation on the future bank rate, which eventually translates to all yields in the market. Nonetheless, it is becoming increasingly apparent that the health of financial and banking institutions can have an effect on economic variables. The advent of the recent financial crisis has thus strengthened the relation between financial and economic variables, rendering macro-finance models of great importance. This section analyzes the recent developments in the use of term structure models of interest rates to macroeconomic and financial applications.

The most natural account of a macroeconomic model is the Taylor rule, which accounts for fluctuations in short rates by using the output gap and inflation gap which are the dispersion of actual levels of output and inflation, respectively, from their target values. \cite{piazzesi_taylor_2007} estimate a Taylor rule and are able to draw the monetary policy shocks by imposing cross-sectional restrictions. An interesting attempt of a macro model is made by \cite{aruoba_2010}, who model the yield curve using level, slope and curvature factors as well as observable macroeconomic variables, amongst which are monetary policy tools, inflation and real activity. \cite{depooter_2010} have a similar approach by analyzing the effect of the inclusion of macroeconomic variables on the forecasting of the term structure of interest rates. Reported results suggest that accounting for macroeconomic informational content improves the forecasting of yields. 

On the finance end of the spectrum, \cite{campbell_2003} examine the interrelation between the expected excess returns on bonds and equity and find that changes in these expected excess returns, real yields and risk levels bear a predictable component. Similarly, \cite{lettau_2011} expand upon this idea by jointly pricing the term structure of interest rates, the risk-return levels of stocks and the returns on the aggregate market.

A recently popular extension of the term structure literature consists in shedding some light on the following twofold research questions. Does the yield curve span yields' volatility, or is volatility unspanned? Those inquisitions have been triggered by a very common phenomenon in the term structure literature, that is the inability of models to jointly capture the first and second moment of yields. \cite{andersen_2007} examine whether bonds do span the yield volatility and find arguments against this hypothesis. Their conclusion was supported by the fact that yield volatility factors were uncorrelated to the yields' cross-section. According to \cite{joslin2_}, volatility is said to be unspanned if bonds are unable to hedge the volatility risk. On this front, it is found that current unspanned stochastic volatility models cannot capture the cross-section of bond volatility. 
Moreover, \cite{giannone_} assess whether macroeconomic content has a predictive ability on the yield curve and on excess bond returns. The use of macroeconomic variables is extended to both the obtention of yield curve factors and the identification of the sources of risk which are not hedged by bonds. Therefore, spanned and unspanned stochastic volatility is a potentially prolific strand of the term structure literature which necessitates further investigation and requires further advances in the years to come. 

Interesting extensions to term structure models can be found in the two types of vector autoregression (VAR) models that follow. The first consists in studying term structure models in a global scale, in the spirit of \cite{diebold_2008} that fits the yield curve of multiple countries by featuring global and country-specific factors. Similarly, \cite{pesaran_2014} introduce the Global VAR model (GVAR). This paper studies the joint forecasting of financial and macroeconomic variables at an international level. Advances in the literature are expected to be made on the selection and number of global factors and individual factors. Additional consideration ought to be made on the existence of regional factor structures. An alternative is to use a Bayesian VAR (BVAR) \`a la \cite{carriero_2011}. This paper,with the help of artificial data, uses a term structure model as a prior. This approach allows the loose imposition of no-arbitrage conditions whilst further alleviating the dimensionality problem and accounting for possible model misspecifications.   


\section{Conclusion}

This chapter provides a brief and concise outline of term structure models, covering basic concepts and introducing several advances within this literature. The general idea that transcends within the chapter is the complexity involved in estimating the term structure of interest rates as well as their potency in extracting information regarding macroeconomic and financial variables. The two following chapters will utilize term structure models in order to extract risk premia. Specifically, chapter 3 emphasizes on the link between term structure models and currencies whilst chapter 4 concentrates on the strong relationship between the yield curve and inflation. Both chapters emphasize on the affine class of term structure models and more specifically on a Nelson-Siegel affine term structure model which further imposes no-arbitrage conditions to ensure the consistency of yield dynamics. 


%\subsection{}