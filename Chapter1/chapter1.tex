\section{Introduction}

The distribution of wealth and income has recently made a comeback to the
center of economic discourse in advanced economies. The ongoing rise of
income inequality, observed since the early 1980s especially in anglo-saxon
countries, has received renewed attention in the public sphere since the
financial started taking its toll on living standards across the world. At
the same time, the bestselling book by \citet{Piketty2014} led to a surge
in interest in the role of capital in the economy, and, by extension, the
distribution of wealth, both in the academic literature and the popular
press. \\
While the broader public has only recently picked up on the issues arising
around income and wealth distributions, they have sat squarely in the center
of many subfields of economics for a long time. The income distribution has
long been of interest to labour economists trying to understand the forces
shaping the evolution of earnings in the labour market, while at least the
accumulation of aggregate wealth plays a central role in macroeconomic
models of economic growth. This chapter, as well as the chapter
\ref{learning}, focuses on the intermediate step that takes us from an
income to a wealth distribution - economic models of household saving.
When attempting to build a model of the wealth distribution, the first
step of course is to get an understanding of the object we want to model.
To this end, this chapter starts by presenting stylised facts of the wealth
distributions in advanced countries and discusses some of the limitations
of the data available. It then builds a simple life-cycle model of consumption
and savings to guide the following discussion and fix notation. Using this
basic model, different savings motives and their importance in the context
of aggregate wealth accumulation are discussed. Following this, the role of
income uncertainty and market structure is examined in more detail.


\section{Stylised facts}
The most noteable and consistent fact that emerges when looking at wealth
distributions across all countries and different time periods is that wealth
is highly unevenly distributed, much more so than income. This can be readily
seen when comparing Gini indices and top shares of income and wealth, as in
Table \ref{gini_topshares} or looking at histograms and cdfs of wealth and
income distributions as in Figure \ref{hist_cdfs}. It is important to note
that in producing these aggregate numbers and figures, one necessarily has
to make decisions on how exactly to construct measures of income and wealth,
which will have to be kept in mind when comparing model predictions with
empricial numbers. When constructing an income measure, the obvious starting
point are labour earnings, and for many economic applications simply
observing an individual's wages will be enough. When thinking about questions
of consumption smoothing though, we are ultimately interested in accounting
for all claims on consumption goods available to an individual or household
in a given period, and while for most people the largest part of these claims
stem from labour earnings, we will also want to account for the effect of
government programs (by constructing a post-tax, post-transfer income measure),
income from accumulated assets, and potentially even informal insurance
arrangements such as inter-vivos transfers between family and friends.
In constructing wealth measures, we again have to think closely about the
question we are trying to answer when construcing them. If the goal is to
account for all productive capital in the economy that can be used in
production, a measure of total net wealth aggregating all forms of asset and
debt classes, and including some durable consumption goods such as cars. When
thinking about the role of wealth in helping the household to smooth out
income fluctuations, it might be more appropriate to exclude very illiquid
assets such as housing, and look more closely at the role of debt for households
which might be at their borrowing constraint and are thus vulnerable to
reductions in their borrowing limit, even though their net wealth (including
illiquid assets) is positive. Finally, important questions are raised by
the existence of various government and private pension schemes, which have
to be factored in when constructing measures of a household's lifetime
resources, but whose exact value might be uncertain (for the case of defined
contribution plans) and not well understood by households themselves.


\section{A workhorse model}
The basic model underlying the discussion of savings behaviour and wealth accumulation
in this chapter is the life-cycle model of household behaviour dating back to
\citet{ModiglianiBrumberg1954} \footnote{A more detailed treatment of the general
class of models can be found both in \citet{BrowningCrossley2001} and in
\citet{AttanasioWeber2010}, although both papers have their focus on household
consumption behaviour rather than wealth accumulation.}. The model can be
written as a single household solving the problem

\begin{align}
\max{c_{t+j}}_{j=0}^{T-t} \delta^{t+j} \mathbb{E}_t \big[ u(c_{t+j}, z_{t+j}) \big] \\ \ref{maxprob}
\intertext{subject to} \\
a_{t+1} = (1+r)(a_t + y_t - c_t)
\end{align}
where $T$ is the last period of the planning horizon, $\delta$ is the subjective
discount factor, $u()$ is the instantaneous felicity function, $c$ is consumption,
$a$ are financial assets, which allow the household to transfer resources across
time, $r$ is a one-period interest rate, and $y$ is income.


\section{Saving motives}

\section{Income uncertainty and market structure}

\section{The role of housing}

\section{Closed and open economies}
On the endogeneity of interest rates

\section{Conclusion}
This chapter provides an overview of stylised facts about the wealth distibutions
in a number of advanced economies and presents various approaches to build economic
models which can account for these stylised facts.
It became clear that while saving for retirement is the main driver of wealth
accumulation for large parts of the population, other factors need to be taken
into consideration to explain the tails of the distribution and the behaviour of
young households. Crucial aspects of an economic model of the wealth distribution
are the risks households are facing -- both on the income and the expenditure side
-- and the financial markets available to them to insure themselves against those
risks and earn returns on their savings. Finally, the far right tail of the wealth
distribution seems to be driven by factors beyond this and it is not clear whether
variation in subjective discount factors, entrepreneurial activity, or other
factors are best suited to model wealth accumulation at the top.
