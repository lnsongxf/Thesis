% This is the heavily changed CUEDthesis class by raffa, november 2011
% CUEDthesis v1.1
% By Harish Bhanderi <harish.bhanderi@cantab.net
% Version 1.0 released 15/07/2002
% Version 1.1 released 14/07/2010
%-------------------------- identification ---------------------
%\NeedsTeXFormat{LaTeX2e}
\ProvidesClass{CUEDthesisPSnPDF}[2010/07/14 v1.1 CUED thesis class]
%-------------------------- initial code -----------------------
\newif\ifCU@bookmode
\CU@bookmodetrue

\DeclareOption{report}{\CU@bookmodefalse}
\DeclareOption{book}{\CU@bookmodetrue}
\ProcessOptions\relax%

\ifCU@bookmode
\DeclareOption*{\PassOptionsToClass{\CurrentOption}{book}}%
\ProcessOptions\relax%
\ifx\pdfoutput\undefined%
   \LoadClass[dvips, a4paper]{book}%
\else%
   \LoadClass[pdftex, a4paper]{book}%
\fi%
\else
\DeclareOption*{\PassOptionsToClass{\CurrentOption}{report}}%
\ProcessOptions\relax%
\ifx\pdfoutput\undefined%
  \LoadClass[dvips, a4paper]{report}%
\else%
   \LoadClass[pdftex, a4paper]{report}%
\fi%
%\renewcommand{\refname}{References}%
\fi

%\DeclareOption{book}%
\usepackage{setspace}
\usepackage{tocbibind}
\usepackage{amssymb}
%\usepackage{graphicx}
\usepackage{fancyhdr}
\usepackage{eucal}
\usepackage[english]{babel}
\usepackage[usenames, dvipsnames]{color}
%\usepackage[perpage]{footmisc}
\usepackage[round]{natbib}
\usepackage{ifthen}
\usepackage{ifpdf}


%Bibliography
%uncomment next line to change bibliography name to references for Book document class
\renewcommand{\bibname}{References}
% note that this doesn't do much if you later define another bibliography style 


% Nomenclature
\usepackage{nomencl}
\makenomenclature
\renewcommand\nomgroup[1]{%
  \ifthenelse{\equal{#1}{A}}{%
   \item[\textbf{Roman Symbols}] }{%             A - Roman
    \ifthenelse{\equal{#1}{G}}{%
     \item[\textbf{Greek Symbols}]}{%             G - Greek
      \ifthenelse{\equal{#1}{R}}{%
        \item[\textbf{Superscripts}]}{%              R - Superscripts
          \ifthenelse{\equal{#1}{S}}{%
           \item[\textbf{Subscripts}]}{{%             S - Subscripts
	    \ifthenelse{\equal{#1}{X}}{%
	     \item[\textbf{Other Symbols}]}{{%    X - Other Symbols
	    \ifthenelse{\equal{#1}{Z}}{%
	     \item[\textbf{Acronyms}]}%              Z - Acronyms
              			{{}}}}}}}}}}

\ifpdf
%-->
%--> Google.com search "hyperref options"
%--> 
%--> http://www.ai.mit.edu/lab/sysadmin/latex/documentation/latex/hyperref/manual.pdf
%--> http://www.chemie.unibas.ch/~vogtp/LaTeX2PDFLaTeX.pdf 
%--> http://www.uni-giessen.de/partosch/eurotex99/ oberdiek/print/sli4a4col.pdf
%--> http://me.in-berlin.de/~miwie/tex-refs/html/latex-packages.html
%-->
    \usepackage[ pdftex, plainpages = false, pdfpagelabels, 
                 pdfpagelayout = useoutlines,
                 bookmarks,
                 bookmarksopen = true,
                 bookmarksnumbered = true,
                 breaklinks = true,
                 linktocpage,
                 pagebackref,
                 colorlinks,%
   		  citecolor=black,%
                 filecolor=black,%
                 linkcolor=black,%
                 urlcolor=black
                 % anchorcolor = black,
                 % hyperindex = true,
                 % hyperfigures
                 ]{hyperref} 
    \usepackage[pdftex]{graphicx}
    \DeclareGraphicsExtensions{.png, .jpg, .pdf}

    \pdfcompresslevel=9
    \graphicspath{{ThesisFigs/PNG/}{ThesisFigs/PDF/}{ThesisFigs/}}
\else
    \usepackage[ dvips, 
                 bookmarks,
                 bookmarksopen = true,
                 bookmarksnumbered = true,
                 breaklinks = true,
                 linktocpage,
                 pagebackref,
                 colorlinks = true,
                 linkcolor =  black,
                 urlcolor  = black,
                 citecolor =  black,
                 %anchorcolor = black,
                 hyperindex = true,
                 hyperfigures
                 ]{hyperref}

    %\usepackage{epsfig}
    \usepackage{graphicx}
    \DeclareGraphicsExtensions{.eps, .ps}
    \graphicspath{{ThesisFigs/EPS/}{ThesisFigs/}}
\fi

%define the page size including an offset for binding
%\setlength{\topmargin}{0.0in}
%\setlength{\oddsidemargin}{0in}
%\setlength{\evensidemargin}{0in}
%\setlength{\textheight}{700pt}
%\setlength{\textwidth}{500pt}


%A4 settings
\ifpdf
   \pdfpageheight=297mm
   \pdfpagewidth=210mm
\else
   \setlength{\paperheight}{297mm}
   \setlength{\paperwidth}{210mm}
\fi

\setlength{\hoffset}{0.00cm}
\setlength{\voffset}{0.00cm}

\setlength{\evensidemargin}{4cm}
%\setlength{\oddsidemargin}{-0.54cm}
\setlength{\topmargin}{1mm}
\setlength{\headheight}{1.36cm}
\setlength{\headsep}{1.00cm}
\setlength{\textheight}{20.84cm}
\setlength{\textwidth}{14.5cm}
\setlength{\marginparsep}{1mm}
\setlength{\marginparwidth}{3cm}
\setlength{\skip}{2.36cm}

\pagestyle{fancy}
\renewcommand{\chaptermark}[1]{\markboth{\MakeUppercase{\thechapter. #1 }}{}}
\renewcommand{\sectionmark}[1]{}
\fancyhf{}
\fancyhead[RO]{\bfseries\rightmark}
\fancyhead[LE]{\bfseries\leftmark}
%\fancyfoot[C]{\thepage}
\renewcommand{\headrulewidth}{0.5pt}
%\renewcommand{\footrulewidth}{0pt}
\addtolength{\headheight}{0.5pt}
\fancypagestyle{plain}{
  \fancyhead{}
  \renewcommand{\headrulewidth}{0pt}
}

\newcommand{\submittedtext}{{A thesis submitted for the degree of}}
%
%
% DECLARATIONS
%
% These macros are used to declare arguments needed for the
% construction of the title page and other preamble.

% The year and term the degree will be officially conferred
%\def\degreedate#1{\gdef\@degreedate{#1}}
% The full (unabbreviated) name of the degree
\def\degree#1{\gdef\@degree{#1}}
% The name of your college or department(eg. Trinity, Pembroke, Maths, Physics)
\def\collegeordept#1{\gdef\@collegeordept{#1}}
% The name of your University
\def\university#1{\gdef\@university{#1}}
% Defining the crest
%\def\crest#1{\gdef\@crest{#1}}

% These macros define an environment for front matter that is always 
% single column even in a double-column document.

\newenvironment{alwayssingle}{%
       \@restonecolfalse\if@twocolumn\@restonecoltrue\onecolumn
       \else\newpage\fi}
       {\if@restonecol\twocolumn\else\newpage\fi}

%define title page layout
\renewcommand{\maketitle}{%
\begin{alwayssingle}
    %\renewcommand{\footnotesize}{\small}
    %\renewcommand{\footnoterule}{\relax}
    \thispagestyle{empty}
%  \null\vfill
  \begin{center}
    { \Huge {\bfseries {\@title}} \par}
{\large \vspace*{35mm} {{\@crest} \par} \vspace*{25mm}}
    {{\Large \@author} \par}
{\large \vspace*{1ex}
    {{\@collegeordept} \par}
\vspace*{1ex}
    {{\@university} \par}
\vspace*{25mm}
    {{\submittedtext} \par}
\vspace*{1ex}
    {\it {\@degree} \par}
\vspace*{2ex}
    {\@degreedate}}
  \end{center}
  \null\vfill
\end{alwayssingle}}

% DEDICATION
%
% The dedication environment makes sure the dedication gets its
% own page and is set out in verse format.

\newenvironment{dedication}
{\begin{alwayssingle}
  \pagestyle{empty}
  \begin{center}
  \vspace*{1.5cm}
  {\LARGE }
  \end{center}
  \vspace{0.5cm}
  \begin{quote} \begin{center}}
{\end{center} \end{quote} \end{alwayssingle}}


% Declaration
%
% The acknowledgements environment puts a large, bold, centered 
% "Acknowledgements" label at the top of the page. The acknowledgements
% themselves appear in a quote environment, i.e. tabbed in at both sides, and
% on its own page.

\newenvironment{declaration}
{\pagestyle{empty}
\begin{alwayssingle}
\begin{center}
\vspace*{1.5cm}
{\Large \bfseries Declaration}
\end{center}
\vspace{0.5cm}
\begin{quote}}
{\end{quote}\end{alwayssingle}}

% The acknowledgementslong environment puts a large, bold, centered 
% "Acknowledgements" label at the top of the page. The acknowledgement itself 
% does not appears in a quote environment so you can get more in.

\newenvironment{declarationlong}
{\pagestyle{empty}
\begin{alwayssingle}
\begin{center}
\vspace*{1.5cm}
{\Large \bfseries Declaration}
\end{center}
\vspace{0.5cm}\begin{quote}}
{\end{quote}\end{alwayssingle}}

\newenvironment{acknowledgements}
{\pagestyle{empty}
\begin{alwayssingle}
\begin{center}
\vspace*{1.5cm}
{\Large \bfseries Acknowledgements}
\end{center}
\vspace{0.5cm}
\begin{quote}}
{\end{quote}\end{alwayssingle}}

% The acknowledgementslong environment puts a large, bold, centered 
% "Acknowledgements" label at the top of the page. The acknowledgement itself 
% does not appears in a quote environment so you can get more in.

\newenvironment{acknowledgementslong}
{\pagestyle{empty}
\begin{alwayssingle}
\begin{center}
\vspace*{1.5cm}
{\Large \bfseries Acknowledgements}
\end{center}
\vspace{0.5cm}\begin{quote}}
{\end{quote}\end{alwayssingle}}

%ABSTRACT
%
%The abstract environment puts a large, bold, centered "Abstract" label at
%the top of the page. The abstract itself appears in a quote environment,
%i.e. tabbed in at both sides, and on its own page.

\newenvironment{abstracts} {\begin{alwayssingle} \pagestyle{empty}
  \begin{center}
  \vspace*{1.5cm}
  {\Large \bfseries  Abstract}
  \end{center}
  \vspace{0.5cm}
   \begin{quote}}
{\end{quote}\end{alwayssingle}}

%The abstractlong environment puts a large, bold, centered "Abstract" label at
%the top of the page. The abstract itself does not appears in a quote
%environment so you can get more in.

\newenvironment{abstractslong} {\begin{alwayssingle} \pagestyle{empty}
  \begin{center}
  \vspace*{1.5cm}
  {\Large \bfseries  Abstract}
  \end{center}
  \vspace{0.5cm} \begin{quote}}
{\end{quote}\end{alwayssingle}}



%ROMANPAGES
%
% The romanpages environment set the page numbering to lowercase roman one
% for the contents and figures lists. It also resets
% page-numbering for the remainder of the dissertation (arabic, starting at 1).

\newenvironment{romanpages}
{\setcounter{page}{1}
  \renewcommand{\thepage}{\roman{page}}}
{\newpage\renewcommand{\thepage}{\arabic{page}}}
%{\newpage\renewcommand{\thepage}{\arabic{page}}\setcounter{page}{1}}

