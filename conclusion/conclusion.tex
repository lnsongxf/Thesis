Throughout this thesis, an outline of affine term structure models is provided. This particular class of term structure models has been made very popular in recent years due to its ability to capture the dynamics of yields both across their time series and cross-section and its ease in imposing the absence of arbitrage, allowing in turn the obtention of adaptable risk premia specifications. Affine term structure models have the advantage of allowing various extensions, in a wide range, to their basic primary setup, asserting their importance in the literature. However, difficulties do arise in their estimation and in the interpretation of the latent factors used. This thesis addresses both problems by utilizing a specific structure to the factor loadings, known as the Nelson-Siegel method. The estimation of this term structure model not only circumvents the global optimum issues but further provides some interpretation to the factors, given the level, slope and curvature factors of the Nelson-Siegel interpolation are not only intuitive in their nature, but also have reliable macroeconomic links.

The present thesis introduces and employs dynamic term structure models to macroeconomic and financial research questions. More precisely, this study initially pertains to financial markets by establishing a tie between interest rates and exchange rates. The study follows by concerning itself with macroeconomic objectives, by exploiting the relationship between yields and inflation.

In a first instance, this study exploits a theoretical relationship between interest rates and exchange rates, namely the uncovered interest rate parity, with the aim to extract currency risk premia through a bilateral affine term structure model with stochastic volatility. The method proposed consists of developing an affine Arbitrage-Free class of dynamic Nelson-Siegel term structure models (AFNS) with stochastic volatility to obtain the domestic and foreign discount rate variations, which in turn are used to derive a representation of exchange rate depreciations. The manipulation of no-arbitrage restrictions allows to endogenously capture currency risk premia. The estimation exercise comprises of a state-space analysis using the Kalman filter. The imposition of the Dynamic Nelson-Siegel (DNS) structure allows for a tractable and robust estimation, offering significant computational benefits, whilst no-arbitrage restrictions enforce the model with theoretically appealing properties. Empirical findings suggest that estimated currency risk premia are able to account for the forward premium puzzle. 

In a second instance, inflation expectations and inflation risk premia are derived using a shadow rate class of term structure models. In response to the recent financial crisis, the Bank of England reduced short term interest rates to 0.5\%. With such low short term rates, traditional term structure models are likely to be inappropriate for estimating inflation expectations and risk premia, because expectations based on such models might implicitly violate the zero lower bound condition. In this segment both the nominal and real UK term structure of interest rates are studied, using the dynamic term structure model introduced by, which imposes the non-negativity of nominal short maturity rates. Estimates of the term premia, inflation risk premia and market-implied inflation expectations are provided. Findings indicate that the zero lower bound specification is necessary to reflect countercyclicality in nominal term premia projections and that medium and long term inflation expectations have been contained within narrower bounds since the early 1990s, suggesting monetary policy credibility after the introduction of inflation targeting.

For my future research projects, I wish to draw from the analysis and discussion of this thesis and elaborate further on this strand of the literature, this time emphasizing on the joint effect of monetary economics and finance on asset prices, financial markets and monetary policy. Two main perspectives emerge within my research agenda. A potential project, that inclines more towards macroeconomic concepts, consists in building an extension of the two above-mentioned models by constructing a Taylor rule type of model which would further extend to include growth. Furthermore, an alternative suggests to further exploit the interaction between macroeconomic and financial data to explore a gap in the literature. Specifically, the study includes providing an economic interpretation to the latent factors, used in the state-space representation, by venturing towards macro-finance models and high frequency data. This analysis is built on the prior belief that assets are affected by macroeconomic conditions but simultaneously suffer from microstructure phenomena. 

Notwithstanding the extensions listed above, it is crucial to note that the most important message to draw from this thesis is that the literature on risk premia is still at its infancy due to the striking complexity involved in estimating an unobservable variable which nonetheless contains a very rich informational content. In turn, in future research, I wish to investigate the sensitivity of the price of risk, and consequently of risk premia, to different specifications in the mean reversion matrix of the states' dynamics. The aim is to determine a preferred specification for dynamic term structure models using a Bayesian shrinkage estimation approach.

To conclude, this thesis builds a spherical account of the versatility of affine models by implementing them to distinct monetary finance applications. Several of the pending issues in the literature are addressed and the grounds for future interesting questions are paved. 

