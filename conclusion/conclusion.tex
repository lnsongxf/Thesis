This thesis considered the effects of heterogeneity in economic models both 
theoretically and empirically. It reviewed the recent literature on the 
macroeconomics of household consumption and savings with agents that differ in 
their economic situation because of only partially insurable shocks to their
idiosyncratic income. After presenting the workhorse Aiyagari-Bewley-Imrohoroglu-
Huggett model of consumption under uncertainty, various extensions to the model
and their effects on the model's predictions are discussed. Amongst these are
the introduction of additional assets such as housing, the consideration of 
differing rates of return either because of different financial assets or because
of entrepreneurial activity, an overlapping generations structure with bequest
motives and institutional factors such as asset-based means testing for public
insurance program. 

After noting that all models of the consumer wealth distribution rely on a 
parsimonious AR(1) process with a persistent and transitory shock component, 
the thesis moves on to consider the recent literature on estimating income 
processes from the variance-covariance structure of earnings residuals. While
this literature has a long tradition, recent work has renewed interest in the 
estimation of processes in which the stochastic process for income features
a deterministic trend component which varies across households. Chapter 
\ref{ch:income} adds to this literature by considering the so far longest sample
of the US PSID to estimate such processes, analyse their variation over time 
by considering sub-periods of the full sample and for the first time estimating
these processes from British data from the BHPS. It documents substantial 
heterogeneity in the estimates obtained for different time periods and income
definitions, although a common theme in the obtained estimates is that both the 
persistence of the AR(1) component and the variance of its innovation are
significantly lower than those estimated
from processes without deterministic growth rate heterogeneity. 

Building on these findings, chapter \ref{ch:learning} then used a life-cycle
model of household saving based on an heterogeneous income process in which 
households learn about their individual specific intercept and slope parameter
over the course of their working life to simulate wealth distributions. It 
showed that while learning is not important for the qualitative predictions 
of the model regarding the shape of the wealth distribution, the key drivers
in model fit are the persistence of the AR(1) process and its innovation variance,
precisely those parameters that in chapter \ref{ch:income} were estimated to 
be significantly lower under a HIP specification. The basic problem in fitting 
the model to the data is that the HIP process assigns a large part of the 
variability in household earning to the dispersion in growth rates, and a lower
part to the permanent shock component. As permanent growth, in contrast to 
persistent shocks, does not require asset accumulation for consumption smoothing,
the lifetime income inequality created by inequality in deterministic growth 
rates, does not lead to the large inequality in wealth holdings that inequality
created by a volatile and persistent shock component in income does. As recent
work points towards an HIP process as a good description of the actual income 
risk facing households, the results of models offering a good fit to the wealth
distribution based on a persistent AR(1) component with large innovation variance
have to be questioned.

In chapter \ref{ch:trade}, the thesis then considers the effects of heterogeneity
on the supply side of the economy by testing the predictions of a model of 
international trade based on firms differing in their marginal productivity levels.
Based on a sample of 64 industries in the three NAFTA member countries Canada,
Mexico and USA, error correction models relating changes in the growth rate
of tariff barriers, firms, and market size to changes in the growth rate of relative
prices, productivities and markups are being estimated on country pairs. In an 
extension of the previous literature, the chapter also considers the effects of
market entry by constructing measures of firm turnover for each industry and 
analysing samples of high- and low turnover industries separately, and considers
the effects of third country tariff barriers on two trade partners. It finds
support for the model's prediction regarding the effect of entry condition, 
namely industries with free entry displaying a larger reaction to changes in 
trade freeness in the long run, as well as a faster speed of adjustment. The
effects of third country tariffs are absent in most specifications, a result
that is likely to be due to the unfortunately less than complete tariff data 
available for some parts of the sample. 

Throughout the thesis it has become clear that heterogeneity has important effects
on the predictions of economic models and is crucially important when applying 
these models for policy analysis. At the same time, heterogeneity can be added
in a variety of ways, many of which often help to explain similar patterns in the
data. Here it is important to consider the effects of different dimensions of 
heterogeneity simultaneously, to understand their interplay and avoid ascribing
too large a role quantitatively to one specific mechanism. As computer power 
continues to grow, more and more complex models of differences between economic
agents will become feasible to solve, making heterogeneity in economics a fruitful
area for future research. 